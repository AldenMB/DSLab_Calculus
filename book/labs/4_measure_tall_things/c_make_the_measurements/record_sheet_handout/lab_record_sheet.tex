\documentclass{article}
\usepackage[italicdiff]{physics}
\usepackage{tikz}
\usepackage{tabularx}    
\usepackage{geometry}
\newgeometry{vmargin={15mm}, hmargin={12mm,17mm}}  
\usepackage{adjustbox}

\title{Lab record sheet}
\author{Alden Bradford}

\newcommand\pointer[2]{
\adjustbox{valign=c}{
\begin{tikzpicture}
\node(image) at (0,0) { \includegraphics[width=0.33\textwidth]{#1} };
\draw[latex-, very thick,red] #2 -- ++(-0.5,0) node[left,black,fill=white]{\small Measure here};
\end{tikzpicture}
}
}

\def\arraystretch{2} % give more room to write

\newcommand{\datatable}{
\begin{tabularx}{0.6\textwidth}{r|*{8}{X}r}
 & & & \multicolumn{3}{l}{near} & \multicolumn{3}{l}{far}\\
Trial& $k$ & $r$  & $a_x$ & $a_y$ & $a_z$ & $a_x$ & $a_y$ & $a_z$\\\hline
1\\\hline
2\\\hline
3\\
\end{tabularx}}

\begin{document}

\section*{Instructions}
The tables below are meant as a guide, to be sure you are recording all the necessary information. You are welcome to take more notes than this, but at a minimum you should fill every cell in the table. The pictures show the specific spot on the building you should measure to.


\appendix

\section{Armstrong}
\datatable
\pointer{armstrong.jpg}{(-0.9, 1.2)}

\section{Arch}
\datatable
\pointer{arch.jpg}{(0,0.45)}

\section{Engineering Fountain}
\datatable
\pointer{fountain.jpg}{(0.1,1)}

\section{Bell Tower}
\datatable
\pointer{belltower.jpg}{(0.1,1.42)}

\section{Math}
\datatable
\pointer{math.jpg}{(-0.2,0.62)}

\end{document}
